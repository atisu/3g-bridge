\documentclass[a4paper, 12pt]{article} 

\usepackage{t1enc} 
\usepackage[latin2]{inputenc} 
\usepackage{graphics} 
\usepackage{verbatim}

\title{3G Bridge User Manual -- v0.2}
% \resizebox{3cm}{!}{\includegraphics{sztaki.jpg}}
\author{Laboratory of Parallel and Distributed Systems}
\date{}

\begin{document} 
\pagestyle{empty}
\maketitle
\thispagestyle{empty}

\section{3G bridge}

The purpose of the 3G bridge application is to create a connection between a job source and a grid backend. The supported grid backends are BOINC and EGEE. The 3G bridge communicates with the job source via a MySQL database. The job source renders job descriptors in the MySQL database. The 3G bridge processes the job descriptors by creating job packages suitable for the underlying grid backend. These job packages are submitted by the 3G bridge into the grid backend. The grid backend processes the job packages and notifies the 3G bridge of any returning results. The 3G bridge settles the outputs of the returning results on the path predefined by the job source and sets the state of the job to finished. The job source polls the database for any finished jobs and gathers the outputs for them.

\section{Execution}
The 3G bridge is meant to be run as a daemon process. The sole, but mandatory argument taken by the 3G bridge is the name of the configuration file. The configuration file should be named {\tt bridge.conf} and should be placed in the same directory in which the 3G bridge resides. The command below executes the 3G bridge as a daemon process and directing the log messages into the file {\tt bridge.log}.
\begin{verbatim}
nohup ./bridge bridge.conf 2>&1 > bridge.log &
\end{verbatim}
Upon successful execution the 3G bridge can access the database and can initialize the grid backend. After initialization, the 3G bridge will poll the database for jobs to be processed in a varying interval between 1 second and 5 minutes.

The 3G bridge can also be deployed as a BOINC daemon. Refer to section \ref{subsec:boinc} on how to deploy the 3G bridge as a BOINC daemon. 

\section{Grid backends}
\subsection{BOINC}
The 3G bridge should be placed in a separate directory under the home directory of the corresponding BOINC project and should be executed by the project administrator user, to ensure that it has the proper right to access the BOINC database. In this case there is no need to create a separate database for the 3G bridge, but the tables for the 3G bridge can be inserted in the database of the BOINC project.

\subsection{EGEE}
The 3G bridge can be executed with the EGEE backend solely. In this case, you have to create a separate MySQL database with the neccessary tables supplied in the file {\tt schema.sql}. Refer to section \ref{sec:db} on how to create the tables. If the 3G bridge is to be executed with both backends at the same time, the EGEE backend can use the database of the BOINC project. 

\section{Database}
\label{sec:db}

The 3G bridge uses database tables to communicate with the job source These tables has to be created prior to first use of the 3G bridge. A database has to be created for the tables of the 3G bridge or in case of the BOINC backend, the tables can be inserted in the database of the BOINC project. Locate the supplied file named {\tt schema.sql} and execute the command below under the administrator user account of the grid environment, to create the neccessary tables for the 3G bridge.
\begin{verbatim}
mysql <database_name> < schema.sql
Assuming the database name is qm_db, eg.:
mysql qm_db < schema.sql
\end{verbatim}

\subsection{Adding algorithms to the database}

Prior to the first use of the 3G bridge the {\tt cg\_algqueue} table has to be populated with algrithms, so the 3G bridge will know which algorithms to handle. To insert a new algorithm into the {\tt cg\_algqueue} table execute the following command after configuring it to your needs:

\begin{verbatim}
INSERT INTO cg_algqueue (grid, alg, batchsize) VALUES 
(<NAME OF GRID>, <NAME OF EXECUTABLE>, <INT SIZE OF BATCH>);
\end{verbatim}

For example, let us assume that the 3G bridge will handle the following algorithm. The name of the algorithm is "Sample". The name of the executable for the algorithm is "sample\_executable". The name of the grid environment the algorithm will be executed in is "see-grid". The 3G bridge can pack up to 10 jobs in one package handled to the grid environment. The SQL command for the above example is as follows:

\begin{verbatim}
INSERT INTO cg_algqueue (grid, alg, batchsize) VALUES 
("see-grid", "sample_executable", 10);
\end{verbatim}

Note that the 3G bridge is capable of managing the same algorithm in multiple grid environments so the above example can be executed multiple times by changing the value of the grid name only.

\subsection{Removing algortihms from the database}

To remove an algorithm from the database provide the name of the grid and the name of the algorithm to   the following SQL command.

\begin{verbatim}
DELETE FROM cg_algqueue WHERE
grid = "see-grid" and alg =  "sample_executable";
\end{verbatim}

\section{Configuration}

The 3G bridge needs to be properly configured before execution. For this a configuration file conforming with the format given below has to be provided. 

\begin{verbatim}
###############################################################
# Sample 3G bridge configuration file
#
# For the syntax, see the "Basic format of the file" and 
# "Possible value types" sections of the Desktop Entry 
# Specification at
# http://standards.freedesktop.org/desktop-entry-spec/latest/

############################
# Database access parameters
[database]

# Name of the database
name =

# Host name of the database
host =

# Database user
user =

# Database password
password =

# Max. number of database connections that can be 
# opened simultaneously
max-connections = 4

#####################################
# Sample grid using the DC-API plugin
[SZDG]

handler = DC-API
dc-api-config = dc-api.conf

###################################
# Sample grid using the EGEE plugin
[SEE-GRID]

handler = EGEE
voname = seegrid
wmproxy-endpoint = ...
myproxy_host = ...
myproxy_port = ...
myproxy_user = ...
myproxy_pass = ...

\end{verbatim}

The database access parameters are mandatory. {\tt host} is the name of the host running the database, usually {\tt localhost} (setting the value to 0 also means localhost). 

\subsection{Grid backend specific configuration}
\subsubsection{BOINC}
\label{subsec:boinc}
The 3G bridge uses the DC-API to communicate with the BOINC backend. The DC-API needs a separate configuration file. {\tt dc-api-config} parameter is the optional name of the DC-API configuration file. The default name of the file the 3G bridge looks for is {\tt dcapi.conf}. If you would like to place the DC-API configuration data in  another file, than you have to provied the {\tt dc-api-config} parameter. 

The format of the DC-API configuration file is as follows:

\begin{verbatim}
[Master]
#-----------------------------------------------------
# General configuration, for both BOINC implementation
#-----------------------------------------------------

#
# Working directory. boinc_appmgr will set it automatically
#

WorkingDirectory = NONE

#
# UUID Instance. boinc_appmgr will set it automatically
#

InstanceUUID = NONE

#
# Debugging level
#

LogLevel = Debug

#----------------------------------------------
# Configuration for BOINC DC-API implementation
#----------------------------------------------

#
# boinc config xml file. boinc_appmgr will set it automatically
#

BoincConfigXML = NONE

#
# BOINC project root directory. boinc_appmgr will set it 
# automatically
#

ProjectRootDir = NONE

#-------------------------
# Per-client configuration
#-------------------------

[Client-sample]

BatchHeadTemplate =
BatchBodyTemplate =
BatchTailTemplate =

BatchPackScript = batch_pack
BatchUnpackScript = batch_unpack
\end{verbatim}
The parameters of the DC-API configuration file can be set manually, or the 3G bridge can be deployed on the BOINC project with the \\ boinc\_appmgr utility, which also sets master side parameters and deploys the 3G bridge as a BOINC daemon. In this case the 3G bridge can be started and stopped along with the other BOINC daemons using the {\tt start} and {\tt stop} commands of BOINC respectively. 

\subsubsection{EGEE}

EGEE Plugin instances use the following parameters:
\begin{description}
\item[handler] type of the plugin. This should be EGEE for EGEE instances
\item[voname] name of the virtual organization to use when submitting jobs using this instance
\item[wmproxy-endpoint] URL of the WMProxy server. Used during submission
\item[myproxy\_host] hostname of the MyProxy server used to fetch proxies to be used for submission
\item[myproxy\_port] port of the MyProxy server used to fetch proxies to be used for submission
\item[myproxy\_user] username for the MyProxy server used to fetch proxies to be used for submission
\item[myproxy\_pass] password for the MyProxy server used to fetch proxies to be used for submission.
\end{description}

Every EGEE Plugin instance must have a unique name, but different EGEE Plugin instances are allowed to use the same virtual organizations. This is useful for cases when different proxies should be used to submit jobs to a given VO. In this case, two different EGEE Plugin instances can be created using the same \texttt{voname} and \texttt{wmproxy-endpoint}, but different MyProxy settings.

\end{document}
